\documentclass[11pt,preprint, authoryear]{elsarticle}

\usepackage{lmodern}
%%%% My spacing
\usepackage{setspace}
\setstretch{1.2}
\DeclareMathSizes{12}{14}{10}{10}

% Wrap around which gives all figures included the [H] command, or places it "here". This can be tedious to code in Rmarkdown.
\usepackage{float}
\let\origfigure\figure
\let\endorigfigure\endfigure
\renewenvironment{figure}[1][2] {
    \expandafter\origfigure\expandafter[H]
} {
    \endorigfigure
}

\let\origtable\table
\let\endorigtable\endtable
\renewenvironment{table}[1][2] {
    \expandafter\origtable\expandafter[H]
} {
    \endorigtable
}


\usepackage{ifxetex,ifluatex}
\usepackage{fixltx2e} % provides \textsubscript
\ifnum 0\ifxetex 1\fi\ifluatex 1\fi=0 % if pdftex
  \usepackage[T1]{fontenc}
  \usepackage[utf8]{inputenc}
\else % if luatex or xelatex
  \ifxetex
    \usepackage{mathspec}
    \usepackage{xltxtra,xunicode}
  \else
    \usepackage{fontspec}
  \fi
  \defaultfontfeatures{Mapping=tex-text,Scale=MatchLowercase}
  \newcommand{\euro}{€}
\fi

\usepackage{amssymb, amsmath, amsthm, amsfonts}

\def\bibsection{\section*{References}} %%% Make "References" appear before bibliography


\usepackage[round]{natbib}

\usepackage{longtable}
\usepackage[margin=2.3cm,bottom=2cm,top=2.5cm, includefoot]{geometry}
\usepackage{fancyhdr}
\usepackage[bottom, hang, flushmargin]{footmisc}
\usepackage{graphicx}
\numberwithin{equation}{section}
\numberwithin{figure}{section}
\numberwithin{table}{section}
\setlength{\parindent}{0cm}
\setlength{\parskip}{1.3ex plus 0.5ex minus 0.3ex}
\usepackage{textcomp}
\renewcommand{\headrulewidth}{0.2pt}
\renewcommand{\footrulewidth}{0.3pt}

\usepackage{array}
\newcolumntype{x}[1]{>{\centering\arraybackslash\hspace{0pt}}p{#1}}

%%%%  Remove the "preprint submitted to" part. Don't worry about this either, it just looks better without it:
\makeatletter
\def\ps@pprintTitle{%
  \let\@oddhead\@empty
  \let\@evenhead\@empty
  \let\@oddfoot\@empty
  \let\@evenfoot\@oddfoot
}
\makeatother

 \def\tightlist{} % This allows for subbullets!

\usepackage{hyperref}
\hypersetup{breaklinks=true,
            bookmarks=true,
            colorlinks=true,
            citecolor=blue,
            urlcolor=blue,
            linkcolor=blue,
            pdfborder={0 0 0}}


% The following packages allow huxtable to work:
\usepackage{siunitx}
\usepackage{multirow}
\usepackage{hhline}
\usepackage{calc}
\usepackage{tabularx}
\usepackage{booktabs}
\usepackage{caption}


\newenvironment{columns}[1][]{}{}

\newenvironment{column}[1]{\begin{minipage}{#1}\ignorespaces}{%
\end{minipage}
\ifhmode\unskip\fi
\aftergroup\useignorespacesandallpars}

\def\useignorespacesandallpars#1\ignorespaces\fi{%
#1\fi\ignorespacesandallpars}

\makeatletter
\def\ignorespacesandallpars{%
  \@ifnextchar\par
    {\expandafter\ignorespacesandallpars\@gobble}%
    {}%
}
\makeatother

\newlength{\cslhangindent}
\setlength{\cslhangindent}{1.5em}
\newenvironment{CSLReferences}%
  {\setlength{\parindent}{0pt}%
  \everypar{\setlength{\hangindent}{\cslhangindent}}\ignorespaces}%
  {\par}


\urlstyle{same}  % don't use monospace font for urls
\setlength{\parindent}{0pt}
\setlength{\parskip}{6pt plus 2pt minus 1pt}
\setlength{\emergencystretch}{3em}  % prevent overfull lines
\setcounter{secnumdepth}{5}

%%% Use protect on footnotes to avoid problems with footnotes in titles
\let\rmarkdownfootnote\footnote%
\def\footnote{\protect\rmarkdownfootnote}
\IfFileExists{upquote.sty}{\usepackage{upquote}}{}

%%% Include extra packages specified by user

%%% Hard setting column skips for reports - this ensures greater consistency and control over the length settings in the document.
%% page layout
%% paragraphs
\setlength{\baselineskip}{12pt plus 0pt minus 0pt}
\setlength{\parskip}{12pt plus 0pt minus 0pt}
\setlength{\parindent}{0pt plus 0pt minus 0pt}
%% floats
\setlength{\floatsep}{12pt plus 0 pt minus 0pt}
\setlength{\textfloatsep}{20pt plus 0pt minus 0pt}
\setlength{\intextsep}{14pt plus 0pt minus 0pt}
\setlength{\dbltextfloatsep}{20pt plus 0pt minus 0pt}
\setlength{\dblfloatsep}{14pt plus 0pt minus 0pt}
%% maths
\setlength{\abovedisplayskip}{12pt plus 0pt minus 0pt}
\setlength{\belowdisplayskip}{12pt plus 0pt minus 0pt}
%% lists
\setlength{\topsep}{10pt plus 0pt minus 0pt}
\setlength{\partopsep}{3pt plus 0pt minus 0pt}
\setlength{\itemsep}{5pt plus 0pt minus 0pt}
\setlength{\labelsep}{8mm plus 0mm minus 0mm}
\setlength{\parsep}{\the\parskip}
\setlength{\listparindent}{\the\parindent}
%% verbatim
\setlength{\fboxsep}{5pt plus 0pt minus 0pt}



\begin{document}



\begin{frontmatter}  %

\title{What Caused The Early Millenium Slowdown? Evidenece Based on
Vector Autoregressions}

% Set to FALSE if wanting to remove title (for submission)




\author[Add1]{Jessica Van der Berg}
\ead{20190565@sun.ac.za}





\address[Add1]{20190565}


\begin{abstract}
\small{
Abstract to be written here. The abstract should not be too long and
should provide the reader with a good understanding what you are writing
about. Academic papers are not like novels where you keep the reader in
suspense. To be effective in getting others to read your paper, be as
open and concise about your findings here as possible. Ideally, upon
reading your abstract, the reader should feel he / she must read your
paper in entirety.
}
\end{abstract}

\vspace{1cm}

\begin{keyword}
\footnotesize{
Multivariate GARCH \sep Kalman Filter \sep Copula \\ \vspace{0.3cm}
\textit{JEL classification} L250 \sep L100
}
\end{keyword}
\vspace{0.5cm}
\end{frontmatter}



%________________________
% Header and Footers
%%%%%%%%%%%%%%%%%%%%%%%%%%%%%%%%%
\pagestyle{fancy}
\chead{}
\rhead{A replication of Gert Peersman (2005) paper}
\lfoot{}
\rfoot{\footnotesize Page \thepage}
\lhead{}
%\rfoot{\footnotesize Page \thepage } % "e.g. Page 2"
\cfoot{}

%\setlength\headheight{30pt}
%%%%%%%%%%%%%%%%%%%%%%%%%%%%%%%%%
%________________________

\headsep 35pt % So that header does not go over title




\hypertarget{test-wether-variables-are-stationary}{%
\section{\texorpdfstring{Test wether variables are stationary
\label{stationary}}{Test wether variables are stationary }}\label{test-wether-variables-are-stationary}}

Variables that are included in the dataset (same order): oil, output
growth, consumer inflation and short-term nominal interest rate for EU
and US.

Gideon suggested I only do the replication for the US, since this will
be a lot of work.

In order to test whether a variable is stationary, you can use a unit
root test such as the Dickey-Fuller (DF) test

Null hypothesis: There is a unit root Alternative hypothesis: Time
series is stationary

If p-values is less than 0.05, it means we can reject the null
hypothesis.

\includegraphics{replication_files/figure-latex/unnamed-chunk-1-1.pdf}

\begin{verbatim}
## 
##  Augmented Dickey-Fuller Test
## 
## data:  slowdown_dataset$OIL
## Dickey-Fuller = -2.0358, Lag order = 5, p-value = 0.5616
## alternative hypothesis: stationary
\end{verbatim}

\begin{verbatim}
## 
##  Augmented Dickey-Fuller Test
## 
## data:  slowdown_dataset$YUS
## Dickey-Fuller = -1.3894, Lag order = 5, p-value = 0.8304
## alternative hypothesis: stationary
\end{verbatim}

\begin{verbatim}
## 
##  Augmented Dickey-Fuller Test
## 
## data:  slowdown_dataset$SUS
## Dickey-Fuller = -3.4394, Lag order = 5, p-value = 0.05113
## alternative hypothesis: stationary
\end{verbatim}

\begin{verbatim}
## 
##  Augmented Dickey-Fuller Test
## 
## data:  slowdown_dataset$CPUS
## Dickey-Fuller = -1.5413, Lag order = 5, p-value = 0.7672
## alternative hypothesis: stationary
\end{verbatim}

\begin{verbatim}
## 
##  Phillips-Perron Unit Root Test
## 
## data:  slowdown_dataset$OIL
## Dickey-Fuller Z(alpha) = -8.7804, Truncation lag parameter = 4, p-value
## = 0.6091
## alternative hypothesis: stationary
\end{verbatim}

\begin{verbatim}
## 
##  Phillips-Perron Unit Root Test
## 
## data:  slowdown_dataset$YUS
## Dickey-Fuller Z(alpha) = -3.501, Truncation lag parameter = 4, p-value
## = 0.9108
## alternative hypothesis: stationary
\end{verbatim}

\begin{verbatim}
## 
##  Phillips-Perron Unit Root Test
## 
## data:  slowdown_dataset$SUS
## Dickey-Fuller Z(alpha) = -11.903, Truncation lag parameter = 4, p-value
## = 0.4288
## alternative hypothesis: stationary
\end{verbatim}

\begin{verbatim}
## 
##  Phillips-Perron Unit Root Test
## 
## data:  slowdown_dataset$CPUS
## Dickey-Fuller Z(alpha) = -1.0566, Truncation lag parameter = 4, p-value
## = 0.9855
## alternative hypothesis: stationary
\end{verbatim}

\hypertarget{impulse-response-function}{%
\section{Impulse response function}\label{impulse-response-function}}

First thing I need to do is convert the data to a time series object in
R. And to do this I need to create a date column.

The graph below, just shows you the dataset for the US. This is nice
because you can see the pattern all the vaaraibles follow. This is not
in the paper but might be nice to put in under `descriptive statistics'.
\includegraphics{replication_files/figure-latex/unnamed-chunk-2-1.pdf}

\begin{verbatim}
## NULL
\end{verbatim}

\hypertarget{conclusion}{%
\section{Conclusion}\label{conclusion}}

\newpage

\hypertarget{references}{%
\section*{References}\label{references}}
\addcontentsline{toc}{section}{References}

\hypertarget{refs}{}
\begin{CSLReferences}{0}{0}
\end{CSLReferences}

\hypertarget{appendix}{%
\section*{Appendix}\label{appendix}}
\addcontentsline{toc}{section}{Appendix}

\hypertarget{appendix-a}{%
\subsection*{Appendix A}\label{appendix-a}}
\addcontentsline{toc}{subsection}{Appendix A}

Some appendix information here

\hypertarget{appendix-b}{%
\subsection*{Appendix B}\label{appendix-b}}
\addcontentsline{toc}{subsection}{Appendix B}

\bibliography{Tex/ref}





\end{document}
