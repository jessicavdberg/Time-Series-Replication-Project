\documentclass[11pt,preprint, authoryear]{elsarticle}

\usepackage{lmodern}
%%%% My spacing
\usepackage{setspace}
\setstretch{1.2}
\DeclareMathSizes{12}{14}{10}{10}

% Wrap around which gives all figures included the [H] command, or places it "here". This can be tedious to code in Rmarkdown.
\usepackage{float}
\let\origfigure\figure
\let\endorigfigure\endfigure
\renewenvironment{figure}[1][2] {
    \expandafter\origfigure\expandafter[H]
} {
    \endorigfigure
}

\let\origtable\table
\let\endorigtable\endtable
\renewenvironment{table}[1][2] {
    \expandafter\origtable\expandafter[H]
} {
    \endorigtable
}


\usepackage{ifxetex,ifluatex}
\usepackage{fixltx2e} % provides \textsubscript
\ifnum 0\ifxetex 1\fi\ifluatex 1\fi=0 % if pdftex
  \usepackage[T1]{fontenc}
  \usepackage[utf8]{inputenc}
\else % if luatex or xelatex
  \ifxetex
    \usepackage{mathspec}
    \usepackage{xltxtra,xunicode}
  \else
    \usepackage{fontspec}
  \fi
  \defaultfontfeatures{Mapping=tex-text,Scale=MatchLowercase}
  \newcommand{\euro}{€}
\fi

\usepackage{amssymb, amsmath, amsthm, amsfonts}

\def\bibsection{\section*{References}} %%% Make "References" appear before bibliography


\usepackage[round]{natbib}

\usepackage{longtable}
\usepackage[margin=2.3cm,bottom=2cm,top=2.5cm, includefoot]{geometry}
\usepackage{fancyhdr}
\usepackage[bottom, hang, flushmargin]{footmisc}
\usepackage{graphicx}
\numberwithin{equation}{section}
\numberwithin{figure}{section}
\numberwithin{table}{section}
\setlength{\parindent}{0cm}
\setlength{\parskip}{1.3ex plus 0.5ex minus 0.3ex}
\usepackage{textcomp}
\renewcommand{\headrulewidth}{0.2pt}
\renewcommand{\footrulewidth}{0.3pt}

\usepackage{array}
\newcolumntype{x}[1]{>{\centering\arraybackslash\hspace{0pt}}p{#1}}

%%%%  Remove the "preprint submitted to" part. Don't worry about this either, it just looks better without it:
\makeatletter
\def\ps@pprintTitle{%
  \let\@oddhead\@empty
  \let\@evenhead\@empty
  \let\@oddfoot\@empty
  \let\@evenfoot\@oddfoot
}
\makeatother

 \def\tightlist{} % This allows for subbullets!

\usepackage{hyperref}
\hypersetup{breaklinks=true,
            bookmarks=true,
            colorlinks=true,
            citecolor=blue,
            urlcolor=blue,
            linkcolor=blue,
            pdfborder={0 0 0}}


% The following packages allow huxtable to work:
\usepackage{siunitx}
\usepackage{multirow}
\usepackage{hhline}
\usepackage{calc}
\usepackage{tabularx}
\usepackage{booktabs}
\usepackage{caption}


\newenvironment{columns}[1][]{}{}

\newenvironment{column}[1]{\begin{minipage}{#1}\ignorespaces}{%
\end{minipage}
\ifhmode\unskip\fi
\aftergroup\useignorespacesandallpars}

\def\useignorespacesandallpars#1\ignorespaces\fi{%
#1\fi\ignorespacesandallpars}

\makeatletter
\def\ignorespacesandallpars{%
  \@ifnextchar\par
    {\expandafter\ignorespacesandallpars\@gobble}%
    {}%
}
\makeatother

\newlength{\cslhangindent}
\setlength{\cslhangindent}{1.5em}
\newenvironment{CSLReferences}%
  {\setlength{\parindent}{0pt}%
  \everypar{\setlength{\hangindent}{\cslhangindent}}\ignorespaces}%
  {\par}


\urlstyle{same}  % don't use monospace font for urls
\setlength{\parindent}{0pt}
\setlength{\parskip}{6pt plus 2pt minus 1pt}
\setlength{\emergencystretch}{3em}  % prevent overfull lines
\setcounter{secnumdepth}{5}

%%% Use protect on footnotes to avoid problems with footnotes in titles
\let\rmarkdownfootnote\footnote%
\def\footnote{\protect\rmarkdownfootnote}
\IfFileExists{upquote.sty}{\usepackage{upquote}}{}

%%% Include extra packages specified by user

%%% Hard setting column skips for reports - this ensures greater consistency and control over the length settings in the document.
%% page layout
%% paragraphs
\setlength{\baselineskip}{12pt plus 0pt minus 0pt}
\setlength{\parskip}{12pt plus 0pt minus 0pt}
\setlength{\parindent}{0pt plus 0pt minus 0pt}
%% floats
\setlength{\floatsep}{12pt plus 0 pt minus 0pt}
\setlength{\textfloatsep}{20pt plus 0pt minus 0pt}
\setlength{\intextsep}{14pt plus 0pt minus 0pt}
\setlength{\dbltextfloatsep}{20pt plus 0pt minus 0pt}
\setlength{\dblfloatsep}{14pt plus 0pt minus 0pt}
%% maths
\setlength{\abovedisplayskip}{12pt plus 0pt minus 0pt}
\setlength{\belowdisplayskip}{12pt plus 0pt minus 0pt}
%% lists
\setlength{\topsep}{10pt plus 0pt minus 0pt}
\setlength{\partopsep}{3pt plus 0pt minus 0pt}
\setlength{\itemsep}{5pt plus 0pt minus 0pt}
\setlength{\labelsep}{8mm plus 0mm minus 0mm}
\setlength{\parsep}{\the\parskip}
\setlength{\listparindent}{\the\parindent}
%% verbatim
\setlength{\fboxsep}{5pt plus 0pt minus 0pt}



\begin{document}



\begin{frontmatter}  %

\title{What Caused The Early Millenium Slowdown? Evidenece Based on
Vector Autoregressions}

% Set to FALSE if wanting to remove title (for submission)




\author[Add1]{Jessica Van der Berg}
\ead{20190565@sun.ac.za}





\address[Add1]{20190565}


\begin{abstract}
\small{
Abstract to be written here. The abstract should not be too long and
should provide the reader with a good understanding what you are writing
about. Academic papers are not like novels where you keep the reader in
suspense. To be effective in getting others to read your paper, be as
open and concise about your findings here as possible. Ideally, upon
reading your abstract, the reader should feel he / she must read your
paper in entirety.
}
\end{abstract}

\vspace{1cm}

\begin{keyword}
\footnotesize{
Multivariate GARCH \sep Kalman Filter \sep Copula \\ \vspace{0.3cm}
\textit{JEL classification} L250 \sep L100
}
\end{keyword}
\vspace{0.5cm}
\end{frontmatter}



%________________________
% Header and Footers
%%%%%%%%%%%%%%%%%%%%%%%%%%%%%%%%%
\pagestyle{fancy}
\chead{}
\rhead{A replication of Gert Peersman (2005) paper}
\lfoot{}
\rfoot{\footnotesize Page \thepage}
\lhead{}
%\rfoot{\footnotesize Page \thepage } % "e.g. Page 2"
\cfoot{}

%\setlength\headheight{30pt}
%%%%%%%%%%%%%%%%%%%%%%%%%%%%%%%%%
%________________________

\headsep 35pt % So that header does not go over title




\hypertarget{introduction}{%
\section{Introduction}\label{introduction}}

In this research assignment, I replicate a research assignment by Gert
Peersman (2005), a German economist, titled ``What caused the early
millennium slowdown? Evidence based on vector autoregressions''. In this
paper, Peersman (2005) uses a simple four-variable VAR (vector
autogressive model) and an identification based scheme based on sign
restrictions to examine the effects of a supply, demand, monetary policy
and oil price shocks. Peersman (2005) uses data from the United States
and Euro area. However, this assignment will only focus on analyzing
shocks for the USA. Peersman (2005) concludes that the millennial
slowdown is not the result of one particular shock, but a combination of
them. The goal of this assignment is to replicate the results of
Peersman (2005) as well as preform additional robustness test to ensure
the validity of Peermans (2005) results.

\begin{itemize}
\tightlist
\item
  summarize what robustiness checks and analysis you did(including the
  ones that you replicate)
\end{itemize}

This paper is structured as follows: The first section will give an
overview of the paper with respects to the economics, methodology and
data that Peersman (2005) used. The second section will replicate the
results for the US. The third section will perform robustness checks and
the forth section will conclude.

\hypertarget{overview-of-the-paper}{%
\section{Overview of the paper}\label{overview-of-the-paper}}

This section gives a brief overview of the economics of Peersman (2005)
and a critical evaluation of the statistical approach that Peersman
(2005) choose to analyze possible causes for the economic slowdown of
the US.

\hypertarget{theory-of-paper}{%
\subsection{Theory of Paper}\label{theory-of-paper}}

The 1990s was the start of an economic boom for the United States as
they experienced the unusual combination of rapid output growth and
extremely low and stable inflation. From 1994 to 2002 the United States
real GDP grew by an average annual rate of almost 4 percent while annual
inflation was less than 2 percent. However, by the end of 2001, the US
began to experience negative growth (Peersman, 2005). Since 2001, the US
economy has not experienced close to the same economic growth. It is
therefore important to understand what caused the slowdown.

The economic expansion that the US experienced made headlines as it was
the longest expansion in economic history. Many explanations have been
offered as to why the US experienced such rapid growth, but overall, it
is attributed to numerous factors. Productivity growth increased
tremendously which created a favorable investment environment. The
private investment opportunities contributed to the advancements in
technologies and inspired innovation (Weller, 2002). Furthermore, the
Federal Open Market Committee (FOMC) created an environment for the
Federal Reserve to keep inflation rates low and stable (Taylor, 1998).

Another possible reason is that the US economy mostly experienced
positive shocks during the late 1990s. The Fed has the responsibility to
responds to shocks to the economy to ensure that output, employment, and
inflation remain stable. The Fed can easily respond to a demand shock as
it pushed output, employment, and inflation in the same direction.
Therefore, the Fed will lower interest rate which will increase money
supply to combat the effects of a demand shock. However, supply shock,
such as an increase in oil prices, are more complicated to respond. With
the threat of a recession, the Fed will need to decide whether to
prioritize inflation stability or employment stability. Which attributed
the economic boom that the US experienced, is the fact that large supply
shocks were uncommon during the 1990s.

After 10 years of economic growth, the US economy entered a recession.
The 2001/2002 was relatively short lived. Kliesen (2003) argues that the
recession was caused by shocks to investment by businesses and
households and by a decline in real net exports. However, Kleisen (2003)
does recognize that it can be extremely difficult to challenging to
discover the root cause of a recession. Understanding fluctuation in the
growth of an economy has been research by many economists but it is
still relatively poorly understood. Therefore, Peersman (2005) adds
value to research by examining four shocks (oil, demand, supply, and
monetary policy) to analyze the origin of the slowdown.

Peersman (2005) concludes that the slowdown of economic growth was
caused by a combination of several shocks. This conclusion is
non-surprising and therefore Peersman (2005) main contribution is the
impressive mathematical and statistical analysis that he performed.

\hypertarget{data-and-methodology}{%
\subsection{Data and Methodology}\label{data-and-methodology}}

This paper used quarterly data from 1980 to 2002 on the US consumer
expenditure index (CPUS), real GDP (YUS), short term nominal interest
rate (SUS) as well as data on the oil price (OIL). Changes in CPUS
record the rate of consumer inflation while changes in YUS record output
growth. The data is then manipulated to display the first difference of
the log of OIL, YUS and CPUS and these variables are assumed to be I(1)
variables. The SUS variable is taken as an I(0) variable (Ouliaris,
Pagan \& Restrepo, 2018). The data is presented in figure 2.1 below.

\begin{figure}[H]

{\centering \includegraphics{replication_files/figure-latex/Figure1-1} 

}

\caption{Data\label{Figure1}}\label{fig:Figure1}
\end{figure}

Peersman (2005) estimates a four-variable constant-coefficient vector
autoregression (VAR) model, with three lags and identifies four shocks.
The variables included in the analysis is the first difference of oil
prices, output growth measured by the real GDP index, consumer inflation
measured by consumer price index and short-term nominal interest rates.
The four shocks that are analyzed in this paper; two supply shocks, one
demand shock and one monetary policy shock. The two supply shocks
represents a shock to the oil price and a shock to output growth. The
demand shock is associated with a shock to consumer inflation (CPUS),
while the monetary policy shock is associated with a shock to the short
term nominal interest rate. Taking the four varaibles and shocks into
account, the equation can be presented as follows:

\[\begin{bmatrix} \Delta oil_t \\ \Delta y_t \\ \Delta p_t \\ s_t \end{bmatrix} = \Biggl[ I - \sum_{i=1}^{n} A_i \Biggl]^{-1} \begin{bmatrix} b_{11}& b_{12}& b_{13} & b_{14} \\
b_{21}& b_{22}& b_{23} & b_{24} \\
b_{31}& b_{32}& b_{33} & b_{34} \\
b_{41}& b_{42}& b_{43} & b_{44} \end{bmatrix} \begin{bmatrix} \epsilon_t^{oil} \\ \epsilon_t^{s} \\ \epsilon_t^{d} \\ \epsilon_t^{m} \end{bmatrix}\]

Where \(\Delta oil_t\), \(\Delta y_t\) and \(\Delta p_t\) represent the
first difference of the price of oil, output growth and the consumer
prices respectively. \(s_t\) represents the short term nominal interest
rate. The oil price, demand, supply and monetary shock in represented by
\(\epsilon_t^{oil}\), \(\epsilon_t^{d}\), \(\epsilon_t^{s}\) and
\(\epsilon_t^{m}\) respectively.

Peersman (2005) assumes that the variables follow a covariance
stationary process. He uses the Dickey fuller test to reject the null
hypothesis of the existence of a unit root at a 10 percent level for
OIL, YUS and CPUS, however, the null hypothesis for interest rates
cannot be rejected. Peersman (2005) makes the assumption that interest
rates are stationary since the nominal rate cannot have a unit root if
both the real rate and inflation are stationary. This assumption will be
intensely analyzed in section 4 during which I will preform numerous
robustness checks.

The paper makes use of a traditional identification strategy using a
combination of short-run and long-run restrictions. Peersman (2005)
assumes that there is a contemporaneous impact of an oil shock on all
variables in the VAR, but no immediate impact of other shocks on oil
prices. This assumption is consistent with previous literature. Further
Peersman (2005) adds the restrictions that a monetary policy shock has
no contemporaneous effect on output, since monetary policy shocks have a
temporary effect on output. To model these contemporaneous effects
\(b_{12} = b_{13} = b_{14} = b_{24} = 0\). However, to accurately model
the contempoarous effects, the VAR needs \([k^2-k]/2\) restrictions,
where k represents the number of variables. This implies that a further
two restrictions are needed in order to model the VAR successfully,
which will be long run restrictions. The contemporaneous matrix that
include only short run restrictions is presented below;

\[\ \begin{bmatrix} b_{11}& 0 & 0 & 0 \\
b_{21}& b_{22}& b_{23} & 0 \\
b_{31}& b_{32}& b_{33} & b_{34} \\
b_{41}& b_{42}& b_{43} & b_{44} \end{bmatrix} \]

Peersman (2005) follows Blanchard and Quah (1989), Gali (1992) and
Gerlach and Smeth (1995) to add long-run restrictions to the model. He
assumes that demand shock has a permanent zero long-run effect of output
growth YUS. Furthermore, Peersman (2005) assumes that monetary shocks
has a zero long-run impact on output growth but a non-zero effect of OIL
and CPUS. Therefore, your long-run restriction matrix will be
represented as follow;

\[\ \begin{bmatrix} *& *& *& * \\
*& *& 0 & 0 \\
*& *& *& * \\
*& *& * & * \end{bmatrix} \]

where the zero's represent restrictions. In the next section I present
and evaluate impulse response functions that are correlated with the VAR
specified by Peersman (2005).

\hypertarget{replication}{%
\section{Replication}\label{replication}}

In this section, I use Peersman (2005) data on the US to evaluate the
VAR by plotting impulse response function, 40 periods ahead, with an
84th and 16th percent confidence band. In Peersman (2005) paper, he
applies four short run restrictions to his VAR and two long-run
restrictions, adding up to the necessary 6 restrictions that is needed
for the evaluation. In this section, I examine each variables response
to the different shocks and compare them to Peersman (2005) results
\footnote{Peersman (2005) original results are displayed in appendix A}.

Under the restrictions specified by Peersman (2005), an oil price shock
is interpreted as an shock to oil price (OIL), a demand shock is
interpreted as a shock to the consumer expenditure index (CPUS), a
supply shock is interpreted as a shock to output growth (YUS) and a
monetary policy shock is interpreted as a shock to the short term
nominal interest rate (SUS).

\hypertarget{oils-response}{%
\subsection{Oil's response}\label{oils-response}}

Oil is one of the most important commodities in the world. A lower oil
price benefits consumer as it provides cheaper traveling cost and a
lower price of gasoline. The oil price also impacts the price of many
manufactured goods, therefore it affects every consumer. Thus, it is
important to understand how the oil price will respond to different
types of shocks. Figure 3.1 shows that there is a permanent effect on
the price of oil for all shocks. The oil price increases after a
positive demand shock and decrease for a restrictive monetary policy
shock. A positive oil shock increases the price of oil. These results
coincides with Peersman (2005) results.

There is also a permanent effect on the price of oil for a supply shock,
which contradicts Peersman (2005) results as he found only a temporary
increase in the price of oil after a supply shock. Theoretically, demand
and supply shocks have different impacts on the price of oil, where
demand shocks are usually more persistent with a larger impact than
supply shocks. However, in Figure 3.1, it is observed that they have
similar impacts. Therefore, based on my results and Peersman (2005)
results, I argue that the effects of a supply shock on oil price is
ambiguous.

\begin{figure}[H]

{\centering \includegraphics{replication_files/figure-latex/Figure2-1} 

}

\caption{Response of Oil Price\label{Figure2}}\label{fig:Figure2}
\end{figure}

\hypertarget{output-growth-response}{%
\subsection{Output Growth Response}\label{output-growth-response}}

To implemented the long-run restrictions specified by Peersman (2005) on
output growth, I implement an identification scheme proposed by
Blanchard-Quah (1989). In Blanchard-Quah original model, the assumption
is made that demand shock has no effect on the long run levels of
output. Peersman (2005) makes the same assumption is his papers and also
assumes that monetary policy shocks have no impact on output growth in
the long run. As can be seen in Figure 3.2, my results for the response
of outgrowth for different shocks are an exact replication of Peersman
(2005) results.

Figure 3.2 shows that output growth responds negatively to an oil shock,
and positively to a demand and monetary policy shock. Additionally, a
demand and monetary policy shock only have a temporary effect on output
growth where the effect of a demand shock is longer lasting. Therefore,
a monetary policy shock has a rather insignificant effect on output
growth where as a demand shock boost economic growth.

\begin{figure}[H]

{\centering \includegraphics{replication_files/figure-latex/Figure3-1} 

}

\caption{Response of Output growth\label{Figure3}}\label{fig:Figure3}
\end{figure}

\hypertarget{consumer-expenditure-index-response}{%
\subsection{Consumer Expenditure Index
Response}\label{consumer-expenditure-index-response}}

Figure 3.3 shows the response of consumer inflation to all four shocks.
These results are identical to Peersman (2005) results except for the
effect of a monetary policy shock on consumer prices. An oil shock and a
demand shock has a strong, positive, long run effect on consumer prices.
This makes theoretical sense, a demand shock insinuates that consumers
want to consumer more, hence they are willing to pay more and therefore
consumer prices increases. Furthermore, Gao, Kim, and Saba (2014) found
a positive effect of an oil price shock on total CPI which in mainly
attributed to the positive effect on energy intensive CPI. Therefore,
Goa, Kim and Saba (2014) supports Peersman (2005) results.

A supply shock has a strong, negative long-run effect on consumer
inflation. One example of this is when an positive supply shock occurs
due to an increase in money supply. This benefits consumers and
institutions in the short run but as a negative long run effect since
money purchasing power decreases. On the other hand, a monetary policy
shock seems to have no long-run impact on inflation. This result is
concerning as it contradicts Peersman (2005) results of a permanent
negative effect on inflation.

\begin{figure}[H]

{\centering \includegraphics{replication_files/figure-latex/Figure4-1} 

}

\caption{Consumer Expenditure Index\label{Figure4}}\label{fig:Figure4}
\end{figure}

\hypertarget{short-term-nominal-interest-rate-response}{%
\subsection{Short Term Nominal Interest Rate
Response}\label{short-term-nominal-interest-rate-response}}

Figure 3.4 is an exact replication of Peersman (2005) result.

An oil shock increases interest rates temporarily to counterbalance the
inflationary pressure. The demand and supply shock has a positive effect
on short-term nominal interest rates with the effect slowly diminishing
over time. These results are consistent with literature.

\begin{figure}[H]

{\centering \includegraphics{replication_files/figure-latex/Figure5-1} 

}

\caption{Interest Rate Response\label{Figure5}}\label{fig:Figure5}
\end{figure}

\hypertarget{robustness-checks}{%
\section{Robustness checks}\label{robustness-checks}}

After replicating Peersman (2005) paper, I will now apply numerous
diagnostic and robustness test to determine whether a four-variable
SVAR(3) is an accurate representation of the data generating process. I
first analyze the optimal lag length for the model. Then I do this and
that.

\hypertarget{optimal-lag-length}{%
\subsection{Optimal lag length}\label{optimal-lag-length}}

Estimating a lag length for a VAR time series model is a critical
econometric exercise as one runs the risk of over or under estimation
(Liew, 2004). The model specified in the paper uses a lag length of 3.
There are four different criteria's that one can analyze to decided on
an optimal lag length: the Akaike Information Criterion (AIC), the
Schwarz Criterion (SC), the Hannan Quinn (HQ) and the Final Prediction
Error (FPE).

As can be seen in the table below, Peersman (2005) chose the FPE lag
criteria to choose on an optimal lag length. The table alwso shows that
there is a large discrepancy between the VAR(10) selected by the AIC and
the VAR(1) and VAR(3) selected by the other criterion. This result is
not unusual as AIC usually chooses a larger number of lags. Therefore,
it is preferred to not use the AIC criteria when choosing a number a
lags.

\begin{center}
\begin{tabular}{ |c|c|c|c| } 
 \hline
 AIC(n) & HQ(n) & SC(n) & FPE(n) \\ 
 \hline
 10 & 1 & 1 & 3\\ 
 \hline
\end{tabular}
\end{center}

Liew (2004) conducted a study where he critically analyzed which lag
length criteria should be used to determine the VAR lag length.He
concluded that AIC and FPE produce the least probability of under
estimation when compared to the other selection criteria. Since I have
already argued that AIC is not the best lag length criteria for this VAR
model, I agree with Peersman (2020) choice of using FPE to decide on a
lag length of three.

\hypertarget{test-for-stationary}{%
\subsection{\texorpdfstring{Test for Stationary
\label{stationary}}{Test for Stationary }}\label{test-for-stationary}}

Stationarity is a dominant principle in time series analysis. Stationary
implies that the statistical properties of a time series variable do not
change over time. It is important to know whether your variables are
stationary as many of the econometric test that you reply on require
variables to be stationary. To analyze whether the data is stationary, I
plot the autocorrelation function (ACF) for each variable. On the x-axis
you will have the number of lags and on the y-axis you will have your
autocorrelation value. If the variable is stationary, the ACF will
degrade to zero quickly, whereas if they are non-stationary they will
slowly diminish as the number of lags increase. As seen in figure 4.1
below, the ACF plot implies that OIL and YUS are stationary variables
where CPUS and YUS are non-stationary variables.

\includegraphics{ACF.jpg} To further investigate the stationarity of the
dataset, I preform the Augmented Dickey Fuller (ADF)test, with no trend
since we detrended our variables. The ADF test is a unit root test for
stationarity. I want to test for a unit root since a unit root can bring
about an unpredictable result in your time series analysis. The ADF test
null hypothesis is that there is a unit root present, where the
alternative hypothesis is that the time series is stationary. The
results of the ADF test are given in the table below:

\begin{center}
\begin{tabular}{ |c|c|c| } 
 \hline
 variable & ADF statistic & P-value \\ 
 \hline
 OIL & -5.02 & 0.01\\ 
 YUS & -4.50 & 0.01 \\
 CPUS & -3.64 & 0.01 \\
 SUS & -1.09 & 0.2859 \\
 \hline
\end{tabular}
\end{center}

From this table, we can reject the null hypothesis of a unit root being
present for OIL, YUS and CPUS implying that the variables are
stationary, but we fail to reject the null hypothesis for SUS. The ADF
test results need to be interpreted with caution since it has a high
type I error rate. A type I error is the false rejection of the null
hypothesis. Therefore, I also make use of another unit root test, the
Phillips-Perron (PP) test.

The PP test corrects for any serial correlation and heteroskedasiticy by
building onto the ADF results. Although the ADF test are more widely
used, the PP tests are more robust to general forms of
heteroskedasticity in the error term. Therefore, it is an appropriate
statistic to analyze to determine stationarity. The null hypothesis is
that same as the ADF test, that the time series has a unit root, where
the alternative hypothesis would be that the time series is stationary.
The results for the PP test is displayed in the table below.

\begin{center}
\begin{tabular}{ |c|c|c| } 
 \hline
 variable & PP statistic & P-value \\ 
 \hline
 OIL & -66.5 & 0.01\\ 
 YUS & -56.3 & 0.01 \\
 CPUS & -17 & 0.01 \\
 SUS & -7.25 & 0.0632 \\
 \hline
\end{tabular}
\end{center}

Analyzing the results for the PP test, we reach the exact same
conclusion as with the ADF test. Therefore, after plotting the
autocorrelation function and analyzing two unit root test, we reach the
same conclusion that Peersman (2005) reached in his paper. For OIL, YUS
and CPUS we reject the null hypothesis of a unit root and concluded that
the time series variables are stationary, even though the ACF shows
contradicting results for CPUS.

In all three test preformed reach the same conclusion that SUS is
non-stationary which is supported by Peersman (2005) results. However,
Gerlach and Smets (1995) agrue that the nominal rate cannot have a unit
root when both the real rate and inflation are stationary. Therefore,
the non-staionary result is not of concern.

\hypertarget{test-for-heteroscedascity}{%
\subsection{Test for heteroscedascity}\label{test-for-heteroscedascity}}

\hypertarget{test-for-cointegration}{%
\subsection{Test for cointegration}\label{test-for-cointegration}}

We say that two time series are cointegrated if they have similar
trends. We conduct a cointegration test to assess whether the residuals
from regressing one series onto another is stationary. The residuals
will be stationary if the times series are cointegrated. I will use the
Engle-Granger (EG) test, where the null hypothesis is that no
cointegration exist.

each of which is I(1), are not cointegrated.

\hypertarget{granger-causality}{%
\subsection{Granger Causality}\label{granger-causality}}

We would now like to know whether changes in one variable will have a
effect on changes in other variables in the dataset. The granger
causality test is a test done to determine whether on time series
variable is useful in forecasting another. We say that one time series
variables (x) granger-causes another time series variable (y) if it
depends on its own past values as well as the past values of x. For
predicting y, we then need to look at the past values of y and the past
values of x. The null hypothesis for the Granger Causaily test is that
the variable specified does not granger-cause any of the other variables
in the dataset. The results are shown in the table below.

\begin{center}
\begin{tabular}{ |c|c|c| } 
 \hline
 cause variable & F-stat & P-value \\ 
 \hline
 OIL & 2.8941 & 0.002711\\ 
 YUS & 3.5067 & 0.0003864 \\
 CPUS & 6.9537 & 4.293e-09\\
 SUS & 4.4792 & 1.5995-o5 \\
 \hline
\end{tabular}
\end{center}

For all cause variables, we reject the null hypothesis that the
cause-variable does not granger-cause on any of the other time series
variables.Therefore, we can derive the following results:

\begin{enumerate}
\item The hypothesis that the oil price does not influence the real GDP index, the consumer expenditure index and the short term nominal interest rate is rejected. This implies that the oil price does influences the US economy. 
\item The hypothesis that the real GDP index does not influence oil price, the consumer expenditure index and the short term nominal interest rate is rejected. This implies that real GDP has an effect on the US economy as well as the oil price, which greatly affects countries that are net importers of oil. 
\item The hypothesis that the consumer expenditure index does not influence the real GDP index, the oil price and the short term nominal interest rate is rejected. This implies that the US citizens spending patterns and behaviors affect the US economy as well as the oil price. 
\item The hypothesis that the short term nominal interest rate does not influence the real GDP index, the consumer expenditure index and the oil price is rejected. This implies that monetary policy greatly affects the US economy and has an affect on the oil price.
\end{enumerate}

\hypertarget{diagnostic-tests-and-test-statistics}{%
\section{Diagnostic tests and Test
statistics}\label{diagnostic-tests-and-test-statistics}}

The results for diagnostic test for VAR(1), VAR(2) and VAR(3) are
provided in the table below.

Here you look and interpret all the test to determine whether VAR(1) is
too restrictive. ARGUE this as part of your robustness test for the
paper.

\hypertarget{conclusion}{%
\section{Conclusion}\label{conclusion}}

\newpage

\hypertarget{references}{%
\section*{References}\label{references}}
\addcontentsline{toc}{section}{References}

Blanchard, O.J., 1989. A traditional interpretation of macroeconomic
fluctuations. The American Economic Review, pp.1146-1164.

Gao, L., Kim, H. and Saba, R., 2014. How do oil price shocks affect
consumer prices?. Energy Economics, 45, pp.313-323.

Gerlach S, Smets F. 1995. The monetary transmission mechanism: evidence
from the G7 countries. CEPR Discussion Paper, 1219.

Jiménez-Rodríguez*, R. and Sánchez, M., 2005. Oil price shocks and real
GDP growth: empirical evidence for some OECD countries. Applied
economics, 37(2), pp.201-228.

Khan, M. U. H. (2008). Short run effects of an unanticipated change in
monetary policy: Interpreting macroeconomic dynamics in Pakistan. State
Bank of Pakistan Working Paper, 22.

Kliesen, K.L., 2003. The 2001 recession: How was it different and what
developments may have caused it?.Review-Federal Reserve Bank of Saint
Louis, 85(5), pp.23-38.

Liew, V.K.S., 2004. Which lag length selection criteria should we
employ?. Economics bulletin, 3(33), pp.1-9.

Ouliaris, S., Pagan, A. and Restrepo, J., 2016. Quantitative
macroeconomic modeling with structural vector autoregressions--an EViews
implementation. IHS Global, 13.

Peersman, G., 2005. What caused the early millennium slowdown? Evidence
based on vector autoregressions. Journal of Applied Econometrics, 20(2),
pp.185-207.

Taylor, J.B., 1998. Monetary policy and the long boom. Federal Reserve
Bank of St.~Louis Review, 80(6), p.3.

Weller, C., 2002. Lessons from the 1990s: Long‐term growth prospects for
the US. New Economy, 9(1), pp.57-61.

\hypertarget{appendix}{%
\section*{Appendix}\label{appendix}}
\addcontentsline{toc}{section}{Appendix}

\hypertarget{appendix-a}{%
\subsection*{Appendix A}\label{appendix-a}}
\addcontentsline{toc}{subsection}{Appendix A}

\begin{figure}
\centering
\includegraphics{Capture.jpg}
\caption{Peersman orignial results}
\end{figure}

\hypertarget{appendix-b}{%
\subsection*{Appendix B}\label{appendix-b}}
\addcontentsline{toc}{subsection}{Appendix B}

\bibliography{Tex/ref}





\end{document}
